\documentclass[12pt,a4paper]{article}

\usepackage[english,greek]{babel}
\usepackage[utf8]{inputenc}
\usepackage[T1]{fontenc}

\usepackage{mathtools}
\usepackage{physics}
\usepackage{siunitx}
\usepackage{cancel}
\usepackage{gensymb}

\usepackage{graphicx,caption}
\usepackage{longtable}
\usepackage{appendix}
\graphicspath{ {images/} }
\allowdisplaybreaks

%% Διαστάσεις
\textheight=672pt % 592pt + 80pt
\voffset=-40pt    % -80/2pt
\textwidth=470pt  % 390pt+80pt
\hoffset=-40pt    % -80/2pt

%% Γλώσσες
\newcommand{\en}{\selectlanguage{english}}
\newcommand{\gr}{\selectlanguage{greek}}

%% Μαθηματικοί τελεστές κ.λ.
\DeclareMathOperator{\di}{d\!}
\newcommand*\eval[3]{\left.#1\right\rvert_{#2}^{#3}}
\newcommand{\pe}{\text{\gr{}p}}

\begin{document}

%% Εξώφυλλο
\begin{titlepage}
\gr{}
\centering
Σχολή \\
Τμήμα \\
\today
\vfill
\vfill
{\huge Τίτλος\par}
{\Large Μάθημα\par}
\vfill
Ον/επώνυμο \\
Αρ. μητρώου \\
Ομάδα
\vfill
\vfill
\end{titlepage}

%% Κείμενο
\pagenumbering{arabic} % αρίθμιση σελιδών
\renewcommand\thesubsection{$(${}\alph{subsection}$)$} % αρίθμιση subsections
\renewcommand{\textdexiakeraia}{} % αφαίρεση του "ʹ" (δεξιά κεραία) 
\addto\captionsgreek{\renewcommand{\figurename}{\gr{}Διάγραμμα}} % 'Διαγράμματα'

\section{Ερώτηση.}

Απάντηση.

%% Παράρτημα
%\clearpage
%\renewcommand\thefigure{\thesection.\arabic{figure}} % αρίθμιση
%\setcounter{figure}{0}                               % εικονών
%\newcommand{\appendixpagenumbering}{                   % χρειάζεται
%    \pagenumbering{arabic}                             % για νέα
%    \renewcommand{\thepage}{\thesection-\arabic{page}} % αρίθμηση
%}                                                      % σελιδών
%\begin{appendices}
%\renewcommand\thesection{\Alph{section}}
%\section{Κάποιο Παράρτημα}
%\appendixpagenumbering  % χρειάζεται σε κάθε καινούριο section
%\clearpage
%\section{Κάποιο Άλλο Παράρτημα}
%\appendixpagenumbering
%\clearpage
%\end{appendices}

%% Βιβλιογραφία
%\pagenumbering{gobble} % αφαίρεση αρίθμισης σελιδών
%\begin{thebibliography}{1}
%\bibitem{αναφορά}
%\end{thebibliography}

% Σχήμα
%\begin{figure}[!htb]
%    \captionsetup{width=0.8\textwidth}
%    \centering
%    \includegraphics[width=0.618\textwidth]{example-image-b}
%    \caption{Εικόνα}
%    \label{fig1:image}
%\end{figure}

% Πίνακας
%\begin{longtable}{|c|c|c|}
%    \captionsetup{width=0.8\textwidth}
%    \caption{Πίνακας}\label{tab1:table} \\
%    \hline
%    Α & Β & Γ \\
%    \hline
%    \endhead
%    \hline
%    \endfoot
%    1 & 2 & 3 \\
%    1 & 2 & 3 \\
%\end{longtable}
\end{document}
