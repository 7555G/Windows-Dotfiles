\documentclass[12pt,a4paper]{article}
\usepackage[greek,english]{babel}
\usepackage[utf8]{inputenc}
\usepackage[T1]{fontenc}
\usepackage{mathtools}
\usepackage{physics}
\usepackage{graphicx}
\usepackage{longtable}
\usepackage{appendix}
\graphicspath{ {images/} }

%% Διαστάσεις
\textheight=672pt % 592pt + 80pt
\voffset=-40pt    % -80/2pt
\textwidth=470pt  % 390pt+80pt
\hoffset=-40pt    % -80/2pt

%% Γλώσσες
\newcommand{\en}{\selectlanguage{english}}
\newcommand{\gr}{\selectlanguage{greek}}

\begin{document}

%% Εξώφυλλο
\begin{titlepage}
\gr
\centering
Σχολή\\
Τμήμα\\
\today
\vfill
\vfill
{\huge Τίτλος\par}
{\Large Μάθημα\par}
\vfill
Ον/επώνυμο\\
Αρ. μητρώου\\
Ομάδα
\vfill
\vfill
\end{titlepage}

%% Κείμενο
\pagenumbering{arabic}
\section{\gr Ερώτηση.}
\gr Απάντηση.

%% Παράρτημα
%\newpage
%\newcommand{\appendixpagenumbering}{
%    \pagenumbering{arabic}
%    \renewcommand{\thepage}{\thesection-\arabic{page}}
%}
%\begin{appendices}
%\renewcommand\thesection{\Alph{section}}
%\section{\en some appendix}
%\appendixpagenumbering
%\newpage
%\section{\gr παράρτημα}
%\appendixpagenumbering
%\newpage
%\end{appendices}

%% Βιβλιογραφία
%\pagenumbering{gobble}
%\gr
%\begin{thebibliography}{1}
%\bibitem{αναφορά}
%\end{thebibliography}

% Σχήμα
%\begin{figure}[h]
%    \centering
%    \includegraphics[width=0.618\textwidth]{example-image-b}
%    \caption{\en an image}
%    \label{fig1:image}
%\end{figure}

% Πίνακας
%\begin{longtable}{| c | c | c | c | c |}
%    \caption{Πίνακας}\label{tab1:table} \\
%    \hline
%    Α & Β & Γ & Δ & Ε \\
%    \hline
%    \endhead
%    \hline
%    \endfoot
%    1 & 2 & 3 & 4 & 5 \\
%    1 & 2 & 3 & 4 & 5 \\
%\end{longtable}
%\end{document}
