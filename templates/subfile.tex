\documentclass[report.tex]{subfiles}

\begin{document}

\section{Ερώτηση}

\subsection{Υποερώτηση.}

Απάντηση.

\begin{align*}
    \displaystyle % δεν επιτρέπει στα στοιχεία να συρρικνωθούν πολύ
    \nabla^2\varphi                         &= 0 \\
    \pd[2]{\varphi}{x} + \pd[2]{\varphi}{y} &= 0
\end{align*}

Τέλος.


%%% Λοιπά Στοιχεία %%%%%%%%%%%%%%%%%%%%%%%%%%%%%%%%%%%%%%%%%%%%%%%%%%%%%%%%%

%% Μαθηματικά
%\begin{align*}
%    \displaystyle % δεν επιτρέπει στα στοιχεία να συρρικνωθούν πολύ
%    \nabla^2\varphi                         &= 0 \\
%    \pd[2]{\varphi}{x} + \pd[2]{\varphi}{y} &= 0
%\end{align*}

%% Σχήμα
%\begin{figure}[!htb]
%    \captionsetup{justification=centering, width=0.8\textwidth}
%    \centering
%    %\input{images/epslatex-image-name}
%    \includegraphics[width=0.618\textwidth]{raster-image-name}
%    \caption{Εικόνα}
%    \label{fig1:image}
%\end{figure}

%% Πίνακας
%\begin{longtable}{c c c}
%    \captionsetup{justification=centering, width=0.8\textwidth}
%    \caption{Πίνακας}\label{tab1:table} \\
%    Α & Β & Γ \\
%    \hline
%    \endhead
%    \hline
%    \endfoot
%    1 & 2 & 3 \\
%    1 & 2 & 3 \\
%\end{longtable}

\end{document}
