\documentclass[10pt,landscape]{article}

\usepackage[a4paper, margin=1cm]{geometry}
\usepackage{parskip}
\usepackage{multicol}

% Διαστάσεις
\setlength{\parskip}{6pt}
\setlength{\footskip}{0.5cm}

% Πακέτα για Σχήματα, Πίνκες και Εικόνες
\usepackage{graphicx,caption,subcaption}
\usepackage{longtable}
\usepackage{enumerate}
\usepackage{appendix}
\graphicspath{ {images/} }

% Πακέτα για Μαθηματικά
\usepackage{mathtools}
\usepackage{siunitx}
\usepackage{cancel}

% Γλώσσες
\usepackage[english,greek]{babel}
\usepackage[utf8]{inputenc}
\usepackage[T7,T1]{fontenc}
\newcommand{\gr}{\selectlanguage{greek}}
\newcommand{\en}{\selectlanguage{english}}


%% Μαθηματικοί Τελεστές κ.α. %%%%%%%%%%%%%%%%%%%%%%%%%%%%%%%%%%%%%%%%%%%%%%%

% Ορθό Διαφορικό %%%%%%%%%%%%%%%%%%%%%%%%%%%%%%%%%%%%%%%%%%%%%%%%%%%%%%%%%%%
\makeatletter
\providecommand*{\di}%
    {\@ifnextchar^{\DIfF}{\DIfF^{}}}
\def\DIfF^#1{%
    \mathop{\mathrm{\mathstrut d}}%
        \nolimits^{#1}\gobblespace}
\def\gobblespace{%
    \let\DiffSpace\!%
    \ifx\diffarg(%
        \let\DiffSpace\relax
    \else
        \ifx\diffarg[%
            \let\DiffSpace\relax
        \else
            \ifx\diffarg[%
                \let\DiffSpace\relax
            \fi\fi\fi\DiffSpace}
%%%%%%%%%%%%%%%%%%%%%%%%%%%%%%%%%%%%%%%%%%%%%%%%%%%%%%%%%%%%%%%%%%%%%%%%%%%%

% Εντολές για Θέσιμο Νέων Τελεστών
\providecommand{\newoperator}[3]{\newcommand*{#1}{\mathop{#2}#3}}
\providecommand{\renewoperator}[3]{\renewcommand*{#1}{\mathop{#2}#3}}

% Οι Παρενθέσεις Μικραίνουν
\delimitershortfall-1sp

% Απόλυτο
\providecommand*{\abs}[1]{\ensuremath{\left\lvert#1\right\rvert}}

% Λεκτικός Εκθέτης
\providecommand*{\ap}[1]{\ensuremath{^\mathrm{#1}}}

% Λεκτικός Δείκτης
\providecommand*{\ped}[1]{\ensuremath{_\mathrm{#1}}}

% Παράγωγος
\providecommand*{\od}[3][]{\frac{\di^{#1}#2}{\di #3^{#1}}}

% Μερική Παράγωγος
\providecommand*{\pd}[3][]{\frac{\partial^{#1}#2}{\partial #3^{#1}}}

% Υπολογιεμένο Ολοκλήρωμα |
\providecommand*{\evala}[3]{\ensuremath{\left.#1\right\rvert_{#2}^{#3}}}

% Υπολογισμένο Ολοκλήρωμα []
\providecommand*{\evalb}[3]{\ensuremath{\left[#1\right]_{#2}^{#3}}}

% Ορθό 'π'
\providecommand*{\pu}{\textrm{\greektext p}}

% Ορθό 'e'
\providecommand*{\eu}{\ensuremath{\mathrm{e}}}

% Ορθό 'j'
\providecommand*{\iu}{\ensuremath{\mathrm{j}}}

% Σωστό Re
\renewoperator{\Re}{\mathrm{Re}}{\nolimits}

% Σωστό Im
\renewoperator{\Im}{\mathrm{Im}}{\nolimits}

%%%%%%%%%%%%%%%%%%%%%%%%%%%%%%%%%%%%%%%%%%%%%%%%%%%%%%%%%%%%%%%%%%%%%%%%%%%%


%% Κείμενο
\begin{document}

% Αλφαβητική Αρίθμιση των Subsections
\renewcommand{\thesubsection}{{}(\alph{subsection})}

% Αφαίρεση του "ʹ" (Δεξιά Κεραία) 
\renewcommand{\textdexiakeraia}{}

% Ονομασίες Διαγραμμάτων, Πινάκων κλπ.
\addto\captionsgreek{
    \renewcommand{\figurename}{Διάγραμμα}
    \renewcommand{\bibname}{Βιβλιογραφία}
    \renewcommand{\refname}{Βιβλιογραφία}
}

% Κείμενο σε Στήλες
\begin{multicols}{3}

\section{Ερώτηση.}

\subsection{Υποερώτηση.}

Απάντηση.

\begin{align*}
    \displaystyle % δεν επιτρέπει στα στοιχεία να συρρικνωθούν πολύ
    \nabla^2\varphi \textpi                 &= 0 \\
    \pd[2]{\varphi}{x} + \pd[2]{\varphi}{y} &= 0
\end{align*}

Τέλος.


%% Παράρτημα
%\begin{appendices}
%
%% Αρίθμιση Εικόνων
%\clearpage
%\renewcommand\thefigure{\thesection.\arabic{figure}}
%\setcounter{figure}{0}
%
%% Νέα Αρίθμιση Σελίδων
%\newcommand{\appendixpagenumbering}{
%    \pagenumbering{arabic}
%    \renewcommand{\thepage}{
%        \thesection-\arabic{page}
%    }
%}
%
%\renewcommand\thesection{\Alph{section}}
%\section{Κάποιο Παράρτημα}
%\appendixpagenumbering  % χρειάζεται σε κάθε καινούριο section
%\clearpage
%\section{Κάποιο Άλλο Παράρτημα}
%\appendixpagenumbering  % χρειάζεται σε κάθε καινούριο section
%\clearpage
%\end{appendices}


%% Βιβλιογραφία
%\pagenumbering{gobble} % αφαίρεση αρίθμισης σελιδών
%\begin{thebibliography}{1}
%\bibitem{Αναφορά.}
%\end{thebibliography}


%%% Λοιπά Στοιχεία %%%%%%%%%%%%%%%%%%%%%%%%%%%%%%%%%%%%%%%%%%%%%%%%%%%%%

%% Μαθηματικά
%\begin{align*}
%    \displaystyle % δεν επιτρέπει στα στοιχεία να συρρικνωθούν πολύ
%    \nabla^2\varphi \textpi                 &= 0 \\
%    \pd[2]{\varphi}{x} + \pd[2]{\varphi}{y} &= 0
%\end{align*}


%% Σχήμα
%\begin{figure}[!htb]
%    \captionsetup{width=0.8\textwidth}
%    \centering
%    \includegraphics[width=0.618\textwidth]{example-image-b}
%    \caption{Εικόνα}
%    \label{fig1:image}
%\end{figure}

%% Πίνακας
%\begin{longtable}{|c|c|c|}
%    \captionsetup{width=0.8\textwidth}
%    \caption{Πίνακας}\label{tab1:table} \\
%    \hline
%    Α & Β & Γ \\
%    \hline
%    \endhead
%    \hline
%    \endfoot
%    1 & 2 & 3 \\
%    1 & 2 & 3 \\
%    \hline
%\end{longtable}

\end{document}
